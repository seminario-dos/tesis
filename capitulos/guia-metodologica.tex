En este capítulo se introducen los actividades que se llevaron a cabo para la extracción de un modelo de rendimiento a partir de una función Lambda, tal y como se hizo en la Sección \ref{sec:experimento-1}. 

\section{Selección del caso de uso}

\begin{singlespace}
\begin{algorithm}[H]
\SetAlgoLined
%\KwResult{Write here the result }
% initialization\;
\tcc{Usuarios expertos podrían saltarse el proceso de instrumentalización y extracción}
\eIf{se cuenta con un modelo \texttt{PCM} existente}{
    utilizar este modelo para simulaciones y pruebas\;
}{
 \eIf{lenguaje de la función está escrita en \texttt{Java}}{
    se puede utilizar la biblioteca(SDK) de \texttt{Kieker}\;
 }{
    implementar integración con \texttt{Kieker} explícitamente\;
 }
 \eIf{plataforma FaaS es \texttt{AWS Lambda}}{
     la integración con \texttt{Kieker} se da por medio de \texttt{JMS}\;
     se pueden utilizar las herramientas de monitoreo de \texttt{AWS}\;
 }{
    \tcc{La integración con Kieker podría ser similar a la propuesta o no}
    implementar integración con \texttt{Kieker} explícitamente\;
    hacer uso de herramientas de medición alternas\; 
 }
}
% \While{While condition}{
%  instructions\;
%  \eIf{condition}{
%   instructions1\;
%   instructions2\;
%   }{
%   instructions3\;
%  }
% }
 \caption{How to write algorithms}
\end{algorithm}
\end{singlespace}

Hoy en día se cuenta con una amplia variedad de opciones tanto en plataformas \emph{FaaS} como en lenguajes de programación que pueden ser utilizados para el desarrollo de funciones Lambda. En la Sección \ref{sec:proveedores-faas} se dan a conocer varias de estas opciones.

La selección del lenguaje de programación se hace especialmente importante si es que se desea seguir el enfoque de extracción y obtención de modelos de rendimiento expuesto en este trabajo. Al momento de escribir este documento, el único lenguaje soportado por Kieker, la herramienta que gestiona la bitácora con los eventos producidos por la función Lambda, es Java.

Esto lo que quiere decir es que actualmente Kieker cuenta con bibliotecas escritas en Java que pueden ser fácilmente incluidas en el proyecto que se está desarrollando y de esta forma lograr una integración con algún servicio de bitácoras de eventos de Kieker. Si bien se puede lograr la integración entre Kieker y otro lenguaje de programación aparte de Java, esta integración deberá ser proporcionada por el implementador de forma explícita. 

La plataforma utilizada para la prueba e instalación de la función Lambda fue AWS Lambda. Tal y como se describió en el Capítulo \ref{cap:manejador-imagenes} y en la Sección \ref{sec:experimento-1}, se hizo uso de herramientas de gestión de bitáboras (CloudWatch) y de monitoreo (AWS X-Ray) para obtener mediciones alternativas de la función que fueron de gran ayuda a la hora de refinar el modelo de rendimiento. En el caso que se desee implementar un caso de uno similar a \emph{Image Handler} en algún otro proveedor de servicios \emph{FaaS} (Sección \ref{sec:proveedores-faas}), se deberá de tomar en consideración las herramientas y particularidades de dicha plataforma para la obtención de mediciones de rendimiento.

El enfoque de trabajo expuesto en este documento sigue una serie de actividades que se consideran útiles para la extracción y obtención de modelos de rendimiento de una función Lambda. Estas actividades fueron adoptadas luego de estudiar la literatura recomendada y, debido a que en este trabajo se está haciendo una exploración inicial a este método de trabajo, se decidió seguir estas actividades de acuerdo al orden recomendado.

Implementadores con mayor experiencia en el modelado y simulación basado en componentes podrían prescindir o modificar partes de las actividades expuestas con el fin de obtener resultados más directos de los aquí expuestos. En particular se podría:
\begin{itemize}
    \item Proporcionar un modelo en PCM \emph{a priori} de una función Lambda y de esta forma saltarse el proceso de instrumentalización y extracción.
    \item Utilizar herramientas de medición alternas para la refinamiento del modelo.
\end{itemize}

\section{Instrumentalización con Kieker}
Se tiene que descargar Kieker de \url{http://kieker-monitoring.net/}, instalarlo y ejecutarlo en una computadora en donde se pueda tener acceso y control para manipular las bitáboras obtenidas. Se recomienda la lectura del manual de usuario de Kieker, disponible en \cite{kieker-user-guide}, para obtener mayores detalles en su instalación y configuración.

Kieker permite varias formas de configuración y ejecución. Debido a que las funciones Lambda se ejecutan en ambientes que son inaccesibles por los implementadores, se optó por configurar Kieker para que escuche los eventos de la función por medio de una cola JMS. La cola JMS y el servicio de Kieker fueron instalados y configurados en una máquina virtual independiente.

En la Sección \ref{sec:image-handler-kieker-pmx} se introduce la programación requerida para generar entradas en una bitádora de Kieker. Se generaron eventos de tipo\\ \texttt{OperationExecutionRecord} que son los tipos de regitros más básicos que se pueden producir en Kieker y, además, se configuró la biblioteca de Kieker para que publicara los eventos de rendimiento a una cola JMS. El servicio de Kieker consumía los eventos de la cola y los registraba en una bitácora local.

\begin{center}
Función Lambda $\rightarrow$ JMS $\rightarrow$ Kieker $\rightarrow$ Bitácora
\end{center}

Se recomienda la lectura del manual de usuario y del tutorial de Kieker disponibles en \cite{kieker-user-guide} y en \cite{kieker-icpe-tutorial-2014} respectivamente para tener mayores detalles sobre otros tipos de eventos que pueden ser generados por la biblioteca de Kieker.

\section{Extracción del modelo con PMX}

