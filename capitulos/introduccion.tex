Los servicios de funciones en la nube (\textit{Function-as-a-Service, FaaS}) representan una nueva tendencia de la computación en la nube en donde se permite a los desarrolladores instalar código, en forma de función, en una plataforma de servicios en la nube y en donde la infraestructura de la plataforma es responsable de la ejecución, el aprovisionamiento de recursos, monitoreo y el escalamiento automático del entorno de ejecución. El uso de recursos generalmente se mide con una precisión de milisegundos y la facturación es por usualmente 100 ms de tiempo de CPU utilizado. 

En este contexto, el ``código en forma de función'' es un código que es pequeño, sin estado, que trabaja bajo demanda y que tiene una sola responsabilidad funcional. Debido a que el desarrollador no necesita preocuparse de los aspectos operacionales de la instalación o el mantenimiento del código, la industria empezó a describir este código como uno que no necesitaba de un servidor para su ejecución, o al menos de una instalación de servidor como las utilizadas en esquemas tradicionales de desarrollo, y acuñó el término \textit{serverless} (sin servidor) para referirse a ello. 

\textit{Serverless} se utiliza entonces para describir un modelo de programación y una arquitectura en donde fragmentos de código son ejecutados en la nube sin ningún control sobre los recursos de cómputo en donde el código se ejecuta. Esto de ninguna manera es una indicación de que no hay servidores, sino simplemente que el desarrollador delega la mayoría de aspectos operacionales al proveedor de servicios en la nube. A la versión de \textit{serverless} que utiliza explícitamente funciones como unidad de instalación se le conoce como \textit{Function-as-a-Service}\cite{DBLP:journals/corr/BaldiniCCCFIMMR17}.

Aunque el modelo FaaS brinda nuevas oportunidades, también introduce nuevos retos. Uno de ellos tiene que ver con el rendimiento de la función, puesto que en este modelo solamente se conoce una parte de la historia, la del código, pero se omiten los detalles de la infraestructura que lo ejecuta. La información de esta infraestructura, su configuración y capacidades es relevante para arquitectos y diseñadores de software para lograr estimar el comportamiento de una función en plataformas FaaS. 

El problema de la estimación del rendimiento de aplicaciones en la nube, como lo son las que se ejecutan en plataformas FaaS y arquitecturas basadas en microservicios, es uno de los problemas que  está recibiendo mayor atención especialmente dentro de la comunidad de investigación en ingeniería de rendimiento de software. Se argumenta que a pesar de la importancia de contar con niveles altos de rendimiento, todavía hay una falta de enfoques de ingeniería de rendimiento que consideren de forma explícita las particularidades de los microservicios\cite{Heinrich:2017:PEM:3053600.3053653}.

Si bien, para FaaS, existen plataformas \textit{open source} por medio de las cuales se pueden obtener los detalles de la infraestructura y de esta manera lograr un mejor entendimiento acerca del rendimiento esperado, estas plataformas cuentan con arquitecturas grandes y complejas, lo cual hace que generar estimación se convierta en una tarea sumamente retadora.

En este trabajo se explora la aplicación de modelado de rendimiento de software basado en componentes para funciones que se ejecutan en ambientes FaaS. Para esto se utilizó la función \emph{Image Handler} (un redimensionador de imágenes) como referencia y, a partir de esta, se generaron cargas de trabajo para recolectar datos del rendimiento en una bitácora(\textit{logs}) de ejecución para luego extraer un modelo de rendimiento a partir de estos datos. Una vez que se cuenta con el modelo, se procedió a su análisis y simulación a fin de evaluar si el modelo generado logra explicar el comportamiento de la función bajo las cargas de trabajo utilizadas y otros experimentos propuestos.

Los resultados obtenidos mediante el proceso anterior fueron sumamente alentadores. Las simulaciones lograron explicar en más de un 95\% los tiempos de respuesta observados en la función de referencia cuando esta se ejecutaba en un ambiente de producción. Esto releva que es posible aplicar el modelado de rendimiento basado en componentes en funciones en la nube y que, gracias a este enfoque, se abren nuevas posibilidades para probar, refinar y predecir el comportamiento de una función en la nube incluso antes de ser instalada en un proveedor de servicios en la nube.

Este estudio está organizado de la siguiente manera: en el capítulo \ref{cap:antecedentes} se presenta un marco conceptual sobre ingeniería de rendimiento de software y trabajos de investigación relacionados con ingeniería de rendimiento de software en aplicaciones en la nube, microservicios y \emph{serverless}. En el capítulo \ref{cap:problema} se define el problema a resolver. En el capítulo \ref{cap:justificacion} se proporciona una justificación del proyecto desde las perspectivas de innovación, impacto y profundidad.La hipótesis de este estudio se expone en el capítulo \ref{cap:hipotesis}.  El objetivo general y los objetivos específicos se plantean en el capítulo \ref{cap:objetivos}. El alcance del proyecto se define en el capítulo \ref{cap:alcance}. Los entregables propuestos se listan en el capítulo \ref{cap:entregables}. Los detalles de la implementación de la función \emph{Image Handler} y del proceso de extracción de un modelo de rendimiento se hace en el capítulo \ref{cap:manejador-imagenes}. En el capítulo \ref{cap:diseno-experimental} se desarrollan los experimientos para validar y comparar los resultados del modelo con lo observado en las invocaciones en la función. La guía metodológica del capítulo \ref{cap:guia-metodologica} condensa los pasos llevados a cabo para reproducir lo implementado en otra función. La tesis concluye en el capítulo \ref{cap:conclusiones}, donde se presentan las conclusiones.


%El alcance del proyecto se define en el capitulo \ref{cap:alcance}. Los entregables que se generarán a partir de esta propuesta se listan en el capítulo \ref{cap:entregables}. La metodología de trabajo se indica en el capítulo \ref{cap:metodologia}. La propuesta concluye en el capítulo \ref{cap:cronograma}, donde se presenta el cronograma de actividades.

%Los métodos de predicción de rendimiento basados en modelos permiten a los arquitectos de software evaluar el rendimiento de los sistemas de software durante las primeras etapas de desarrollo. Estos modelos de predicción se centran en los aspectos relevantes de la arquitectura y de la lógica del negocio, dejando de lado detalles de la infraestructura subyacente. Sin embargo, estos detalles son esenciales para generar predicciones de rendimiento que sean precisas.
%
%Para los ingenieros, es una práctica común simular el modelo de un artefacto antes de construirlo. Modelos de diseños de autos, circuitos electrónicos, puentes, entre otros, son simulados para entender el impacto de decisiones de diseño en varias atributos de calidad de interés como seguridad, consumo de energía o estabilidad. La habilidad de predecir las propiedades de un artefacto basado en su diseño y sin necesidad de construirlo, es una de las características centrales de una disciplina de ingeniería. A partir de esta visión, de lo que se considera una disciplina de ingeniería establecida, se podría decir entonces que la ingeniería de software es apenas una disciplina de ingeniería\cite{Reussner:2016:MSS:3036121}. Esto porque frecuentemente los ingenieros de software carecen del entendimiento del impacto de decisiones de diseño en atributos de calidad como rendimiento o confiabilidad. Como resultado, se intenta probar la calidad del software mediante costosos ciclos de prueba y error.
%
%El no entender el impacto en las decisiones de diseño puede ser costoso y riesgoso. El probar software significa que ya se ha hecho un esfuerzo en su implementación. Por ejemplo, si las pruebas revelan problemas de rendimiento, es muy probable que la arquitectura necesita ser modificada, lo que puede conllevar a costos adicionales. Estos costos surgen debido a que en sistemas de software empresarial un bajo rendimiento es principalmente el efecto de una arquitectura inadecuada que efecto de código.
%
%La ingeniería de rendimiento de software(SPE por sus siglas en inglés) es una disciplina que se centra en incorporar aspectos de rendimiento dentro del proceso de desarrollo de software, con el objetivo de entregar software confiable de acuerdo con propiedades de rendimiento particulares. Los modelos de rendimiento predictivos son una de las herramientas empleadas en SPE. Construidos en las fases tempranas del proceso de desarrollo de software, los modelos ayudan a predecir el rendimiento eventual del software y de esta forma guiar el desarrollo, para eso los modelos de predicción de rendimiento deben capturar todos los componentes relevantes del sistema.
%
%Para aplicaciones de software modernas, esto puede implicar modelar complejas capas tales como máquinas virtuales o \emph{middleware} de mensajería. Componer todos estos modelos puede resultar una tarea costosa e ineficiente. En su lugar, un modelo abstracto de la aplicación se puede construir primero y luego ir agregando los modelos de los componentes del sistema. 
%
%En este trabajo se propone la construcción de un modelo de rendimiento para un sistema que utilice \emph{middleware} orientado a mensajes con el fin de evaluar la influencia en el rendimiento de dicho sistema. Se propone evaluar esta influencia por medio de un ejemplo: tomar una aplicación de referencia con el fin de obtener sus métricas actuales de rendimiento, adaptarla para que utilice \emph{middleware} orientado a mensajes y luego medir su rendimiento y generar un modelo a partir de esto.
