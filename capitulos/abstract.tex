\section*{Abstract}
Cloud Functions represent a new trend in cloud computing in which developers are allowed to install code in a \emph{Function-as-a-Service} (FaaS) platform able to manage provisioning, execution, monitoring and automatic scaling. Currently, despite the increasing popularity of the cloud-based software applications, it is reported there is a lack of performance models to explain the behavior of those applications. The underlying infrastructure in FaaS platforms is hidden from the developers and designers and, since the influence of the infrastructure is unknown, this makes things harder when applying software engineering mechanisms on cloud functions, which could lead to wrong or inaccurate performance estimations. In this study, we explore the use of component-based modeling and simulation in order to generate performance estimations of a cloud function which was exercised using a variety of workloads. A function was instrumented to record performance data related with its invocations in a log file and then, using this log file as an input, we extracted a performance model in a \emph{Palladio Component Model} format in which we ran simulations to validate if the generated model could explain the behavior of the function when in runtime. By using this approach and further tunings in the model, we were able to validate the simulations could explain more than 95\% of the behavior of the function and, that component-based modeling and simulation can be considered a serious option when trying to explaining the behavior of a cloud function.

\newpage