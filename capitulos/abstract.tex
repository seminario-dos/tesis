\section*{Abstract}
Cloud Functions represent a new trend in cloud computing in which developers are allowed to install code in a \emph{Function-as-a-Service} (FaaS) platform able to manage provisioning, execution, monitoring and automatic scaling. Currently, despite the increasing popularity of cloud-based software applications, it is reported that there is a lack of performance models to explain the behavior of those applications. The underlying infrastructure in FaaS platforms is hidden from the developers and designers and, since the influence of the infrastructure is unknown, this makes it difficult to apply software performance engineering approaches on cloud functions, which could lead to wrong or inaccurate performance estimations. In this study, we explore the use of component-based modeling and simulation in order to generate performance estimations of an exemplar cloud function which was exercised using a variety of workloads. A cloud function was both implemented and instrumented to record in a log file performance data related with its invocations and then, using the log file as an input, we extracted a performance model in a \emph{Palladio Component Model} format suitable for running simulations to validate whether the generated model could explain the runtime behavior of the function. Using this approach and further tunings in the model, we were able to validate that the simulations could explain more than 95\% of the behavior of the function and that component-based modeling and simulation can be considered a serious option when trying to explain the behavior of a cloud function.

\newpage