\section[Revisión de literatura]{Revisión de literatura\\\small{Alineado con objetivo específico 1}}
Se pretende identificar los resultados de otros estudios relacionados con el modelado de rendimiento de software, así como retos y oportunidades de investigación que existan en esta área. Las preguntas de investigación inicialmente propuestas son las siguientes:
\begin{itemize}
    \item[\textbf{PI1}] ¿Cuáles enfoques de predicción y medición del rendimiento en sistemas de software basados en componente se han propuesto?
    \item[\textbf{PI2}] ¿Cuáles enfoques de predicción y medición de rendimiento de software se han utilizado para aplicaciones en la nube?
    \item[\textbf{PI3}] ¿Qué retos y oportunidades existen con estos enfoques en la actualidad?
    \item[\textbf{PI4}] ¿Qué herramientas hay disponibles para el modelado de rendimiento de software?
\end{itemize}

\section[Implementación de caso de uso de función en la nube]{Implementación de un caso de uso de función en la nube\\\small{Alineado con objetivo específico 2}}

Identificar un caso de uso de que sea considerado como de referencia o de utilidad común en donde las funciones en la nube hayan mostrado ser un buenas candidatas para su solución. Se procederá a implementar este caso de uso e instalarlo en una plataforma FaaS. Esta será la función que servirá como base para futuros análisis.

\section[Pruebas sobre el caso de uso]{Pruebas sobre el caso de uso\\\small{Alineado con objetivo específico 3}}

\subsection{Diseño experimental}
Planeamiento y definición de las variables por observar de las pruebas de carga.

\subsection{Pruebas de carga}
Diseño, ejecución y análisis de los resultados de las pruebas.

\newpage

%\subsection{Resultados}


%\section[Medición del rendimiento del caso de uso]{Medición del rendimiento del caso de uso\\\small{Alineado con objetivo específico 3}}
%Al caso de uso identificado se le realizarán mediciones de su rendimiento por medio de diferentes cargas de trabajo. Esto permitirá conocer el comportamiento de la función y utilizar los resultados obtenidos en estas mediciones para contrastar resultados posteriores.

\section[Modelado y análisis del rendimiento de la función]{Modelado y análisis del rendimiento de la función\\\small{Alineado con objetivo específico 4}}
Se tomarán los datos generados por las pruebas anteriores para utilizarlas como entrada y, a partir de ellos, realizar una labor de análisis y modelado.

\section[Modelo de rendimiento de la función en la nube]{Modelo de rendimiento de la función en la nube\\\small{Alineado con objetivo específico 5}}
A partir de la función propuesta, pruebas y análisis del rendimiento por medio de herramientas de modelado, se propondrá un modelo que logre caracterizar el comportamiento de la función.

\subsection[Simulaciones sobre el modelo propuesto]{Simulaciones sobre el modelo propuesto}
Para validar el modelo propuesto, se ejecutarán simulaciones sobre él y de esta forma determinar si los resultados de las simulaciones se corresponden con los obtenidos en las pruebas de rendimiento de la función.

\section[Guía metodológica]{Guía metodológica\\\small{Alineado con objetivo específico 6}}
Una vez realizado el estudio, se estructurará una descripción del método utilizado para estimar el rendimiento de una función en la nube. 