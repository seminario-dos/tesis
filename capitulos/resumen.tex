\section*{Resumen}
Las funciones en la nube (\emph{Function-as-a-Service, FaaS}) se presentan como una nueva tendencia de la computación en la nube en donde se permite a los desarrolladores instalar código en una plataforma de servicios en la nube y, en donde la infraestructura de la plataforma es responsable de la ejecución, el aprovisionamiento de recursos, monitoreo y el escalamiento automático del entorno de ejecución. Actualmente, pese a la creciente popularidad de las aplicaciones de software basadas en servicios en la nube, se reporta que se carece de modelos de rendimiento que contribuyan a caracterizar el comportamiento de este tipo de aplicaciones. En FaaS, la infraestructura tecnológica subyacente se oculta por completo de los desarrolladores y diseñadores por lo que, al desconocer de esta influencia, se dificulta la aplicación de enfoques de ingeniería de rendimiento de software y se pueden generar estimaciones erróneas del rendimiento de una función. En este estudio se explora la aplicación de modelado y simulación basado en componentes con el fin de obtener estimaciones sobre el rendimiento de una función en la nube bajo distintas cargas de trabajo. Una función fue instrumentalizada para registrar datos de rendimiento asociados a sus invocaciones en una bitácora para luego, a partir de esta, extraer un modelo de rendimiento en \emph{Palladio Component Model} al cual se le ejecutaron simulaciones para evaluar si el modelo obtenido podía explicar las observaciones reales. Mediante este método y refinamientos en el modelo, se pudo validar que las simulaciones podían explicar en más de un 95\% el comportamiento de la función y que el enfoque de modelado y simulación basado en componentes se posa como una alternativa para explicar el comporamiento de una función en la nube. 

\newpage