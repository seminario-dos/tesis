En el modelo FaaS, la infraestructura tecnológica subyacente se oculta por completo de los diseñadores y desarrolladores, asimismo, la duración de la ejecución de una función en la nube determina el costo del servicio: entre mayor sea la duración, mayor será el costo y viceversa. Si bien se han realizado estudios \cite{8360324} para evaluar factores que influyen en el rendimiento de servicios basados en computación \emph{serverless}, el modelado del rendimiento de este tipo de aplicaciones se sigue presentando como un reto de investigación \cite{Heinrich:2017:PEM:3053600.3053653}. El modelado del rendimiento ha ganado considerable atención en la comunidad de ingeniería de rendimiento en las pasadas dos décadas, principalmente en sistemas de software basados en componentes \cite{Koziolek:2010:PEC:1808359.1808729}. Al finalizar y alcanzar los objetivos de la investigación se podrá contar con un marco de referencia que permitirá:
\begin{itemize}
    \item Tener una función en la nube que servirá como prueba de concepto funcional para la evaluación del rendimiento
    \item Contar un método mediante el cual se pueda analizar el rendimiento de una función en la nube por medio de modelado y simulación.
\end{itemize}


Una vez alcanzados los objetivos, será posible responder a la pregunta:


\textbf{¿Es posible estimar el rendimiento de una función en la nube por medio de modelado y simulación basados en componentes?}