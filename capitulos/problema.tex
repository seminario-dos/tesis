De acuerdo con la revisión de la literatura, se carece de modelos de rendimiento que contribuyan a caracterizar el comportamiento de funciones en la nube alojadas en plataformas FaaS bajo distintas cargas de trabajo. Contar con tales modelos permitiría validar si las funciones en la nube pueden cumplir criterios de calidad de servicio especificados, en una consideración temprana de arquitecturas candidatas para desarrollar un sistema intensivo en software.

En las plataformas FaaS en las que se ejecutan funciones en la nube, la infraestructura tecnológica subyacente se oculta por completo de los desarrolladores y diseñadores. El conocimiento de la influencia de esta infraestructura y su configuración es vital para que los arquitectos de software puedan obtener predicciones significativas del comportamiento de una función pues, al omitirse la influencia que esta tiene, puede conducir a la generación de predicciones erróneas con respecto del rendimiento de una función. Una función que reporte tiempos de respuesta sumamente prolongados o bien la utilización de significativas cantidades de recursos puede generar grandes costos económicos y hasta llegar a ser rechazada por la plataforma FaaS.