En esta tesis se exploró el modelado y la simulación basados en componentes sobre funciones en la nube en plataformas \emph{FaaS}, con el fin de evaluar si la aplicación de este método logra entregar información relevante sobre el comportamiento de la función, la cual pueda servir tanto para razonar sobre el diseño e implementación de la arquitectura del software, como también para tener criterios más informados para predecir el comportamiento en ambientes de producción.

Para la realización de este trabajo, se desarrolló e instaló una función Lambda en el servicio AWS Lambda.  Se utilizó una adaptación del enfoque de ingeniería de rendimiento declarativo descrito en \cite{Walter:2018:TDP:3185768.3185777} ya que, al estar haciendo los primeros pasos en el campo de la ingeniería de rendimiento de software, dicho enfoque permitió automatizar gran parte del proceso.

El principal hallazgo obtenido durante la realización de la investigación reportada en esta tesis es que el modelado y la simulación basados en componentes sí logran caracterizar en gran medida el comportamiento de una función en la nube. Luego de la ejecución de miles de invocaciones a la función en la nube y miles de simulaciones sobre el modelo de rendimiento, se pudo corroborar que los conjuntos de resultados obtenidos en ambos escenarios presentaron tendencias muy similares, lo cual fue comprobado estadísticamente. 

La instrumentalización hecha en el código fuente para soportar el monitoreo de los eventos al interior de la función causó que las simulaciones reportaran una diferencia media de 247ms con respecto del tiempo promedio de los tiempos de respuesta de las invocaciones hechas a la función Lambda.

\subsection{Análisis de resultados}
\paragraph{Experimento \#1} El primer experimento de la Sección \ref{sec:experimento-1} estuvo dirigido a generar un modelo de rendimiento a partir de las bitácoras de la función \emph{Image Handler}. Para esto, se implementó una versión de la función que fue instrumentalizada con la biblioteca de Kieker a fin de generar una bitácora en donde registrar los datos del rendimiento de los eventos asociados a la entrada, obtención de la imagen, redimensionamiento y salida de la función. Luego, se tomó dicha bitácora, que fue pasada a PMX para extraer un modelo de rendimiento en formato PCM. 

Una vez obtenido el modelo de rendimiento, se procedió a crear tres experimentos en donde se realizaron mil invocaciones de redimensionamiento sobre tres grupos de imágenes y, al mismo tiempo, se realizaron refinamientos en los componentes del modelo de rendimiento para reflejar los cambios en la demanda de la función para ejecutar mil simulaciones en \emph{Palladio Workbench}.

Finalmente, se compararon los resultados de las invocaciones reales de redimensionamiento con los de los de las simulaciones, las cuales pudieron caracterizar el comportamiento de la función Lambda con un error máximo del 1\%.

\paragraph{Experimento \#2} El experimento de la Sección \ref{sec:experimento-2} fue planteado para explorar el comportamiento de la función Lambda cuando esta era ejercitada durante un periodo continuo y, al mismo tiempo, evaluar si el modelo de rendimiento obtenido en el experimento \#1 podría explicar algo del comportamiento observado aquí.

Se ejecutaron ráfagas de mil invocaciones de redimensionamiento secuenciales utilizando imágenes aleatorias y otras ráfagas de mil invocaciones utilizando una imagen fija. Las invocaciones se realizaron sobre los tres mismos grupos de imágenes definidos en el experimento \#1.

Como resultado de la ejecución de las mil ráfagas se pudo notar con mayor detalle cómo la función Lambda arranca inicialmente en un estado ``frío'' y, conforme va recibiendo invocaciones, va pasando paulatinamente a un estado ``caliente'', como lo señalaba la literatura consultada. Al contrastar los resultados de este experimento con el modelo de rendimiento obtenido del experimento \#1, se observó que hay una relación entre las simulaciones que reportaron los tiempos de respuesta más prolongados con los tiempos de respuesta de la función cuando se encontraba en estado ``frío''. Asimismo, los tiempos de respuesta promedio de las simulaciones concuerdan con los tiempos de respuesta promedio de la función en estado ``caliente''.

\paragraph{Experimento \#3} Descrito en la Sección \ref{sec:experimento-3}, es una variante del experimento \#2. Se introdujeron tiempos de inactividad entre ráfagas de invocaciones de redimensionamiento y, como en el experimento \#2, se buscó evaluar el comportamiento de la función ante estas ráfagas y la posible relación de este comportamiento con el modelo de rendimiento.

Se ejecutaron ráfagas de cien invocaciones de redimensionamiento y se introdujeron tiempos de inactividad de 10, 20, 40 y 80 minutos entre cada ráfaga. Se utilizaron los mismos tres grupos de imágenes del experimento \#1 en las invocaciones de redimensionamiento.

Conforme se fue incrementando el tiempo de inactividad entre ráfagas, observamos un incremento también en la posibilidad de que la función cayera en un estado ``frío'' y, aunque este comportamiento no fue introducido explícitamente en el modelo de rendimiento, sí se pudo notar una correspondencia entre los tiempos de respuesta de la función Lambda en estado ``frío'' y ``caliente'' en las ráfagas individuales con los resultados de las simulaciones.

\paragraph{Experimento \#4} En este experimento se puso a prueba la herramienta \texttt{SAM CLI} para determinar su confiabilidad para simular y tener algún grado de predicción del comportamiento de una función Lambda en un ambiente de producción (en este caso AWS Lambda). 

Para explorar esto, se realizaron mil invocaciones de redimensionamiento en cada una de las dos plataformas, de manera que las secuencias de imágenes por redimensionar fueran idénticas en ambos casos. Los resultados de las invocaciones demostraron que el comportamiento en \texttt{SAM CLI} resultó ser muy diferente al de AWS Lambda y, al menos para el caso de \emph{Image Handler}, no resultó ser de utilidad para obtener una estimación de la ejecución de la función en un ambiente de producción.

\subsection{Cumplimiento de objetivos}
A continuación se dan a conocer las actividades que se llevaron a cabo para cumplir con los objetivos específicos planteados para esta tesis:

%    \item Revisar el estado del arte de trabajos relacionados con enfoques de predicción y medición del rendimiento en sistemas de software como servicio.\label{sec:obj1}
%    \item Sintetizar un caso de uso de una función en la nube considerada como de referencia, con el propósito de analizar su comportamiento.\label{sec:obj2}
%    \item Elaborar, conforme a un diseño experimental, pruebas de rendimiento sobre el caso de uso seleccionado a fin de obtener datos base.
%    \item Analizar los datos experimentales mediante herramientas de extracción de modelos de rendimiento de software.
%    \item Proponer y validar modelos de rendimiento a partir de los experimentos realizados.        
%    \item Formular una guía metodológica para dar a conocer aspectos de rendimiento en funciones en la nube a partir de la experiencia obtenida.

\paragraph{Objetivo \#\ref{sec:obj1}:} \emph{Revisar el estado del arte de trabajos relacionados con enfoques de predicción y medición del rendimiento en sistemas de software como servicio.}\\
En el Capítulo \ref{cap:antecedentes} se presenta una revisión de literatura de estudios relacionados con el modelado de rendimiento de software, retos y oportunidades de investigación que existen en esta área.

\paragraph{Objetivo \#\ref{sec:obj2}} \emph{Sintetizar un caso de uso de una función en la nube considerada como de referencia, con el propósito de analizar su comportamiento}.\\
Se desarrolló la función Lambda \emph{Image Handler}, que permite el redimensionamiento de imágenes ``al vuelo'' y que es considerada como un caso de uso de referencia en AWS. Los detalles de la implementación de esta función Lambda se abarcan en el Capítulo \ref{cap:manejador-imagenes}.
 
\paragraph{Objetivo \#\ref{sec:obj3}} \emph{Elaborar, conforme a un diseño experimental, pruebas de rendimiento sobre el caso de uso seleccionado a fin de obtener datos base.}\\
En los cuatro experimentos planteados en el Capítulo \ref{cap:diseno-experimental}, se realizaron pruebas de rendimiento utilizando tres grupos de imágenes sobre la función Lambda. Los resultados obtenidos a partir de estas pruebas permitieron obtener datos base sobre el rendimiento de la función y hacer comparaciones con los resultados de las simulaciones posteriores.

\paragraph{Objetivo \#\ref{sec:obj4}} \emph{Analizar los datos experimentales mediante herramientas de extracción de modelos de rendimiento de software.}\\
En las Secciones \ref{sec:estrategia-de-extraccion-de-modelo} y \ref{sec:experimento-1} se expone la estrategia y la aplicación de extracción y obtención de un modelo de rendimiento a partir de datos experimentales. En la Sección \ref{sec:estrategia-de-extraccion-de-modelo} se da a conocer la estrategia general para la extracción y obtención de un modelo de rendimiento, la cual, como ahí mismo se indica es una variante del enfoque de ingeniería de rendimiento declarativo propuesto en \cite{Walter:2018:TDP:3185768.3185777}.

Como parte de las tareas realizadas en el experimento llevado a cabo en la Sección \ref{sec:experimento-1} se:
\begin{itemize}
    \item implementaron variantes de la función Lambda a las cuales se les ejecutaron pruebas de rendimiento, que fueron de mucha utilidad para calibrar el modelo de rendimiento obtenido.
    \item se obtuvieron bitácoras con datos del rendimiento de estas funciones Lambda alternativas y
    \item se analizaron las bitácoras y se extrajo un modelo de rendimiento por medio de la herramienta PMX.
\end{itemize}


\paragraph{Objetivo \#\ref{sec:obj5}} \emph{Proponer y validar modelos de rendimiento a partir de los experimentos realizados.}\\
A partir del modelo de rendimiento obtenido, se inició un ciclo de prueba $\rightarrow$ validación $\rightarrow$ afinamiento. Estas actividades se exponen con mayor detalle en la descripción del experimento de la Sección \ref{sec:experimento-1}.

Se ejecutaron simulaciones sobre el modelo y se validaron los resultados contra los obtenidos en la función Lambda base IM-Simple. Se llevaron a cabo cálculos estadísticos para evaluar cuán similares eran ambos conjuntos de resultados.

En los experimentos de las Secciones \ref{sec:experimento-2}, \ref{sec:experimento-3} y \ref{sec:experimento-4}, se tomó el modelo de rendimiento como referencia para comparar los resultados obtenidos de cada experimento con los resultados de las simulaciones sobre el modelo con el fin de explorar las relaciones entre uno y otro.

\paragraph{Objetivo \#\ref{sec:obj6}} \emph{Formular una guía metodológica para dar a conocer aspectos de rendimiento en funciones en la nube a partir de la experiencia obtenida.}\\
La guía metodológica del Capítulo \ref{cap:guia-metodologica} expone las actividades que se llevaron a cabo para la obtención de un modelo de rendimiento a partir de una función Lambda. Se espera que lo expuesto en esta guía sirva como un punto de partida para explorar las herramientas y los enfoques utilizados en este trabajo y, de esta manera, replicarlos o extenderlos en otros casos de uso.

\subsection{Trabajo futuro}
Conforme se fue desarrollando esta tesis se fueron reconociendo aspectos que ameritan ser profundizados en aras de crear mayor conocimiento el área del modelado y la simulación basados en componentes y arquitecturas o aplicaciones en ambientes en la nube.

La lista que se presenta a continuación sintetiza los temas identificados como relevantes para ser profundizados:
\begin{itemize}
    \item \emph{Estudios sobre el rendimiento de funciones en la nube instaladas en plataformas FaaS \emph{Open Source}}: el trabajo de esta tesis se llevó a cabo en la plataforma AWS Lambda, una plataforma propietaria. Si se replicara este estudio en plataformas FaaS \emph{Open Source}, se podría evaluar si el método aquí presentando puede llevarse a cabo en tales ambientes.
    \item \emph{Estudios sobre ingeniería de rendimiento aplicados en ambientes en la nube en general:} este fue uno de los principales retos reportados en la literatura consultada. A pesar de la popularidad creciente de este modelo de computación, no existen muchos estudios asociados a arquitecturas de software 
    \item \emph{Evaluación del rendimiento de las plataformas FaaS por medio de modelado y simulación}: alternativamente, se reconoce valor en examinar el comportamiento de toda una plataforma FaaS por medo de modelado y simulación. Para llevar a cabo este estudio se podría utilizar plataformas FaaS \emph{Open Source} como Apache OpenWhisk. Tal estudio podría brindar mayores detalles de las particulares del comportamiento de la plataforma FaaS y cómo esta interactúa e impacta con el rendimiento de funciones en la nube.
    \item \emph{Extensión del Palladio Workbench para incluir componentes propios para modelado y simulación de aplicaciones en ambientes en la nube}: \emph{Palladio Workbench} está diseñado para el modelado y simulación de software de uso general, por lo que las particulares de las arquitecturas de software en ambientes en la nube no se incluyen. Estas deben ser incluidas de formas explícita e interpretadas con los motores de simulación y visualización con lo que pone a disposición la herramienta. Nuevos componentes para análisis de arquitecturas en la nube en \emph{Palladio Workbench} contribuirían a promover el estudio de carácterísticas de rendimiento, escalabilidad e integración, áreas en las que se ha reportado que existe una necesidad.
    \item \emph{Desarrollo de herramientas para implementación y prototipado de funciones en la nube que tomen en cuenta las particularidades de la instalación y la ejecución en ambientes de producción}: actualmente las herramientas existentes son competentes pero no brindan mayor información sobre la ejecución de las funciones en un eventual ambiente de producción. Esto es importante porque ayudaría a calibrar aún más los aspectos relacionados con la arquitectura y el rendimiento antes que se llegue a dar la instalación.
\end{itemize}

\subsection{Sobre las herramientas de modelado y simulación utilizadas en este estudio}
Kieker, PMX y \emph{Palladio Workbench} fueron las principales herramientas para generar bitácoras de un sistema, extracción de modelos, simulación y visualización de resultados. De las tres, \emph{Palladio Workbench} es la que cuenta tanto con una trayectoria más amplia en el área de la ingeniería de rendimiento de software así como de un grupo de trabajo más activo que está periódicamente lanzando nuevas versiones del producto. Esta herramienta fue probada en estudios preliminares por el autor y estó contribuyó a familiarizarse en la forma en la que esta trabaja.

La adopción y el uso de Kieker y PMX fue algo totalmente nuevo pero, gracias al hecho de que ambas son herramientas estables y que cuentan con una documentación actualizada, la curva de aprendizaje fue baja y se logró que ambas se pudieran integrar con el código existente.

Estas tres herramientas tienen una orientación más académica que comercial por lo que se recomienda ser precavido en su adopción y uso ya que, a pesar que las universidades que las desarrollan han demostrado mantener un desarrollo activo a lo largo de los años, estas podrían dejar de ser mantenidas o no funcionar bajo un escenario en particular. Se recomienda la lectura de la guía metodológica de la Sección \ref{cap:guia-metodologica} para obtener mayores detalles de la forma sobre la que se utilizaron dichas herramientas en este estudio.

\subsection{Reflexiones personales}
En principio, si bien la propuesta de este estudio se presentaba como un reto, se sabía también que había mucho valor en trabajar en esto y poder aportar a esta área de conocimiento. Durante todo este proceso fui alentado por mi profesor tutor Ignacio Trejos y, gracias a su invaluable colaboración, se pudo dar forma a este documento y reportar los hallazgos obtenidos.

En muchas tareas retadoras que se presentan en la vida es difícil medir su complejidad sin antes trabajar en ellas, considero que la realización de una tesis es un ejemplo de esto: en necesario empezar y ser constante con el fin de ahondar en el tema y resolver los problemas que se van dando.

El camino que he recorrido durante el Programa de Maestría en Computación ha sido fascinante y sumamente enriquecedor. He tenido la oportunidad de aprender, conocer nuevas personas y ser mejor. Esto es algo que voy agradecer por siempre al equipo de trabajo del Programa de Maestría y en especial a los profesores con los que tuve el privilegio de ser alumno.